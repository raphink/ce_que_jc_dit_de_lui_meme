\addchap{Pr\'eface}

\begin{preface}
Il n'est pas une grande librairie où l'on entre aujourd'hui sans trouver,
au milieu d'autres livres traitant pêle-mêle de religion, de cuisine bio et de spiritualité orientale,
des ouvrages sur Jésus de Nazareth, ce rabbin itinérant à qui l'on attribue les plus grands
fantasmes, les plus mystérieuses intrigues.

Pourtant, bien loin de nous informer objectivement, presque tous ces essais et romans
traitent d'un Jésus imaginaire, d'un Jésus fantaisiste et en complet désaccord avec
la description que l'histoire a accepté, à travers les récits des quatre \'Evangiles.

\`A l'heure où, peut-être plus encore qu'à toute autre époque, la personne de Jésus
souffre d'être découverte dans son authenticité historique, ce sermon d'une brûlante actualité
nous ramène à la source de l'identité du Christ: son propre témoignage,
tel qu'il est relaté avec fidélité par les auteurs du Nouveau Testament.

On y découvre un Jésus dont les paroles tranchantes ne tolèrent pas l'image
floue et terne qu'on a pu prêter à sa personne,
et dont les actes dénotent la grandeur et la pureté.

Quatre-vingt années après la prédication de ce sermon par Charles-\'Edouard Babut, Charles Staples Lewis,
l'écrivain britannique auteur notamment des \emph{Chroniques de Narnia}, écrivait dans son livre
\emph{Les fondements du Christianisme}:

\begin{quote}
\quotefont
Un homme ordinaire qui aurait dit le genre de choses que Jésus
a dites ne serait pas reconnu comme un grand maître moral.
Il serait plutôt un fou \ocadr similaire à un homme qui prétendrait être un œuf poché \fcadr ou bien il serait le Satan de l'enfer.
À vous de choisir. Soit que cet homme était, et est, le Fils de Dieu, ou bien Il était fou ou pire encore…
\end{quote}


Quel que soit votre choix, la question de l'identité de Jésus-Christ est majeure. Je prie que cette lecture vous éclaire quant à la personne qu'était réellement Jésus.


\begin{flushright}
Raphaël Pinson\\
Pasteur de l'église évangélique\\
Calvary Chapel Chambéry
\end{flushright}

\end{preface}


