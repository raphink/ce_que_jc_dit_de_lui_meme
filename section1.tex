% Do this automatically
%\clearpage
% Needs improvement for title
\section{I}

\Og{} Quoique je rende témoignage de moi-même, mon témoignage est véritable, car je sais d’où je viens et où je vais.\Fg{} Quel est, mes frères, le témoignage auquel le Seigneur fait allusion dans ces paroles solennelles ? En d’autres termes, qu’est-ce que Jésus-Christ a dit de lui-même ? \ocadr{} Pour répondre d’une manière complète à cette question, pour reproduire intégralement le témoignage de Jésus touchant sa personne, il faudrait citer la moitié de ses discours ; pour le commenter, il faudrait écrire un livre, un livre qui, s’il allait au fond des choses, serait tout ensemble la meilleure exposition et la meilleure apologie de la vérité chrétienne. En raison des limites étroites qui nous sont assignées, nous ne pourrons qu’effleurer les sommets de notre sujet. Pour ne pas lasser votre attention, nous ne citerons pas toujours tout au long les paroles du Maître ; aussi bien vous sont-elles familières ; mais nous ne dirons rien qui ne soit appuyé sur des textes précis et positifs et nous nous appliquerons à rester en deçà de ce qu’ils affirment plutôt que de le dépasser.

Un premier fait s’impose à notre attention\frcolon{} Jésus-Christ est, dans tout le domaine de la Révélation, le seul envoyé de Dieu qui rende témoignage de lui-même. Les autres, comme il convient à des pécheurs chargés d’un si grand message, s’effacent devant la vérité qu’ils annoncent, se sentent petits, faibles, indignes en face de la mission qui leur est confiée. Paul, celui de tous qui, contraint par les circonstances, met le plus en avant sa personnalité, se défend absolument de toute prétention à se prêcher lui-même, et quand il apprend qu’un parti religieux se réclame de son nom, il s’indigne et s’écrie\frcolon{} \Og{} Avez-vous donc été baptisés au nom de Paul?\Fg{}\footverse{{ICo}(1:12-16)} Il en est tout autrement de Jésus. Son enseignement est, au moins en très grande partie, une révélation de sa personne. Il est même l’objet des doctrines qu’il prêche, des devoirs qu’il impose, des sentiments religieux qu’il inspire. Il veut qu’on croie en lui, qu’on s’attache à lui ; être persécuté pour son nom ou persécuté pour la justice, c’est la même chose;\footverse{{Mt}(5:10)} son nom ou sa personne se confond avec la religion, avec la vérité même.\footverse{{Jn}(14:6)} Il n’y a que deux explications possibles à un tel fait\frcolon{} ou bien Jésus est, pour dire le moins, beaucoup moins humble que ne le sont ses prédécesseurs ou ses disciples ; ou bien il y a entre eux et lui une différence vraiment essentielle.

Quelle est cette différence ? Ici, mes frères, nous n’osons pas imiter le vol d’aigle d’un saint Jean, qui d’un seul coup d’aile se transporte au plus haut des cieux et va chercher dans le sein de l’éternité et de la divinité même Celui dont il va raconter l’histoire.
Il convient mieux aux
\quotebox{De même Christ, qui s’est offert une seule fois pour porter les péchés de beaucoup d’hommes, apparaîtra sans péché une seconde fois à ceux qui l’attendent pour leur salut.}{\ibibleverse{He}(9:28)}
besoins et aux habitudes de l’esprit moderne et aussi à la faiblesse de notre foi, de prendre la terre pour point de départ.
Aussi bien, c’est ce que fit Jésus-Christ lui-même.
Le nom qu’il affectionnait le plus est celui de Fils de l’homme, nom dont nous n’avons pas le temps de discuter le sens précis et d’indiquer les origines prophétiques, mais qui, pour tout esprit non prévenu, implique certainement ces deux choses\frcolon{} Jésus est homme, Jésus se sait et se sent, dans son humanité même, distinct du reste de l’humanité et supérieur à elle.
Certes, Jésus est homme\frcolon{} chacune de ses paroles est une confession de son humanité et chacun de ses actes le révèle. Il est homme, on le sent à ses pleurs, à ses souffrances, à ses luttes, à ses tentations, à ses prières ; il est homme, d’une humanité réelle et complète, quant à son corps et quant à son âme ; \Og{} semblable en toutes choses à ses frères\Fg{}, dit un apôtre ; mais il ajoute\frcolon{} \Og{} sans péché\Fg{}.
Sans péché ! ces mots ne retranchent rien à l’humanité de Jésus, car, loin d’être un élément essentiel de la nature humaine, le péché est une infirmité, une dépravation de cette même nature ; mais ils suffisent à creuser un abîme entre Jésus et le reste des hommes, tels que l’expérience nous les fait connaître. Or, l’affirmation apostolique n’est, ici comme ailleurs, que l’écho de celle de Jésus lui-même.
Jésus se savait et s’est dit en mille manières pur de tout péché.
\quotebox{\Og{} Ce ne sont pas ceux qui se portent bien qui ont besoin de médecin, mais les malades. Je ne suis pas venu appeler des justes, mais des pécheurs.\Fg{}}{\ibibleverse{Mc}(2:17)}
Je regarde ce fait comme le plus certain de l’histoire.
Je n’en appelle pas seulement à des déclarations expresses du Seigneur telles que celles-ci\frcolon{} \Og{} Qui de vous me convaincra de péché?\footverse{{Jn}(8:46)} \ocadr{} Le Père ne m’a point laissé seul, parce que je fais toujours ce qui lui est agréable.\footverse{{Jn}(8:29)} \ocadr{} Le Prince de ce monde vient, mais il n’a rien en moi.\Fg{}\footverse{{Jn}(14:30)}
Tout l’ensemble de la vie et des discours de Jésus suppose absolument chez lui la conscience de sa sainteté parfaite, immaculée.
Il ne se met jamais sur le même rang que les pécheurs\frcolon{} ils sont les malades, lui le médecin.\footverse{{Mc}(2:17), \ibibleverse{Mt}(9:12), \ibibleverse{Lc}(5:31)}
Il y a un sentiment qu’il recommande à ses disciples comme étant pour eux le commencement et la condition de toute sainteté et dont il ne leur a jamais donné l’exemple, c’est le repentir. Essayez un moment de supposer qu’il ait eu sujet de répéter pour son propre compte ces mots de la prière qu’il nous a enseignée\frcolon{} \Og{} Père, pardonne-nous nos offenses\Fg{},\footverse{{Mt}(6:12), \ibibleverse{Lc}(11:4)} vous n’ajoutez pas un trait à son image, vous la rendez grimaçante, monstrueuse, impossible ; vous anéantissez l’histoire évangélique tout entière. Chose merveilleuse ! Celui qui propose aux hommes la perfection absolue comme idéal et l’exacte imitation de Dieu comme loi, a la certitude intime qu’il observe lui-même cette loi et qu’il réalise cet idéal.

Jésus ne se sent pas seulement séparé des pécheurs par sa sainteté parfaite ; il se place vis-à-vis d’eux comme leur Maître, leur Législateur, leur Roi, leur Juge, leur Sauveur.
Plus grand qu’Abraham,\footverse{{Jn}(8:58)} que Moïse,\footverse{{Mt}(19:8), \ibibleverse{Mc}(10:5)} que David,\footverse{{Mt}(22:44)} que Salomon,\footverse{{Mt}(12:42), \ibibleverse{Lc}(11:31)} que le temple,\footverse{{Mt}(12:6)} que le sabbat,\footverse{{Mt}(12:8), \ibibleverse{Mc}(2:27-28), \ibibleverse{Lc}(6:5)} il est, je ne fais qu’abréger et résumer ses propres déclarations, il est celui qui accomplit la loi,\footverse{{Mt}(5:17)} celui que les prophètes ont désiré et attendu,\footverse{{Mt}(13:17), \ibibleverse{Lc}(10:24)} il est le propriétaire de ce champ qui est le monde,\footverse{{Mt}(13:44)} l’héritier de cette vigne qui est le royaume de Dieu,\footverse{{Mt}(21:33-41), \ibibleverse{Mc}(12:1-9), \ibibleverse{Lc}(20:9-16)} l’époux de cette élite sacrée de l’humanité qui formera son Église.\footverse{{Mt}(22:1-14)}
C’est lui qui, au dernier jour, rendra à chacun selon ses œuvres,\footverse{{Mt}(16:27)} séparera le blé de l’ivraie,\footverse{{Mt}(13:24-30)} les bons et les méchants;\footverse{{Mt}(13:38)} et ce qui déterminera la sentence des uns et des autres, ce sera d’avoir reçu ou rejeté la parole de Jésus,\footverse{{Jn}(5:24)} d’avoir aimé et secouru ce bon Maître dans la personne de ses disciples,\footverse{{Mt}(25:40)} ou de l’avoir méconnu et méprisé.\footverse{{Mt}(25:41-43)}
Il est le Sauveur de l’humanité perdue,\footverse{{Jn}(12:47)} le bon Berger,\footverse{{Jn}(10:11,14)} la Porte des brebis;\footverse{{Jn}(10:7)} il est le Médiateur entre Dieu et les hommes, \ocadr{} si le mot est de saint Paul,\footverse{{ITim}(2:5)} la pensée est partout dans les Évangiles.
Il est le chemin\frcolon{} d’abord le chemin de Dieu vers les hommes, et ensuite le chemin des hommes vers Dieu.
\quotebox{Jésus lui dit: \Og{} Je suis le chemin, la vérité, et la vie. Nul ne vient au Père que par moi. \Fg{}}{\ibibleverse{Jn}(14:6)}
Quel que soit le bien que votre cœur réclame, Jésus vous convie à le chercher en lui ; \Og{} le trouver, c’est trouver toute chose\Fg{}. Vous avez soif de vérité\frcolon{} Jésus est la vérité,\footverse{{Jn}(14:6)} la lumière du monde;\footverse{{Jn}(8:12)} le voir, c’est voir Dieu.\footverse{{Jn}(14:9)} Vous avez besoin de pardon\frcolon{} Jésus a le pouvoir de vous le donner,\footverse{{Mt}(9:2-6), \ibibleverse{Mc}(2:5-11), \ibibleverse{Lc}(5:20-24), \ibibleverse{Lc}(7:48-50)} car il l’a acquis pour vous ; son sang coule sur la croix pour la rémission de vos offenses. Vous soupirez après le repos\frcolon{} Jésus le promet aux âmes travaillées qui viennent à lui;\footverse{{Mt}(11:28)} il leur donne sa paix, et il ne donne pas comme le monde donne.\footverse{{Jn}(14:27)} Vous demandez le Saint-Esprit, c’est Jésus qui l’envoie;\footverse{{Jn}(14:16)} vous êtes affamés et altérés de vie divine, Jésus donne l’eau vive,\footverse{{Jn}(4:14)} il est lui-même le Pain de vie;\footverse{{Jn}(6:35)} il est le Cep dont la sève nourrit et féconde les sarments.\footverse{{Jn}(15:5)} Si quelqu’un a soif, qu’il vienne à lui et qu’il boive!\footverse{{Jn}(7:37)} Assurément, Jésus veut conduire ses disciples à son Père et à leur Père, à son Dieu et à leur Dieu, mais il ajoute\frcolon{} \Og{} Nul ne vient au Père que par moi.\Fg{}\footverse{{Jn}(14:6)}

À ces déclarations si hautes de Jésus correspondent les sentiments qu’il réclame de ses disciples. Il exige d’eux une confiance absolue, une consécration sans réserve à son service, un amour ; sans bornes, devant lequel tout autre amour s’efface et se transforme presque en haine.\footverse{{Lc}(14:26)}
Il veut qu’ils croient en lui comme en Dieu,\footverse{{Jn}(14:1)} qu’ils prient en son nom, qu’ils baptisent en son nom,\footverse{{Mt}(28:19)} qu’ils l’honorent comme ils honorent le Père.
Encore une fois, quel contraste ne remarquons-nous pas à cet égard entre Jésus et tous les autres envoyés de Dieu!
Paul déchire ses vêtements quand les pauvres païens de Lystre veulent lui rendre les honneurs divins.\footverse{{Ac}(8:14)}
Quand le voyant de l’Apocalypse veut se prosterner devant l’ange de la révélation, celui-ci le relève et lui dit\frcolon{} \Og{} Je ne suis que ton compagnon de service, adore Dieu.\Fg{}\footverse{{Ap}(19:10)}
Jésus, lui, n’arrête et ne censure ni la foule qui crie\frcolon{} \Og{} Hosanna au fils de David!\Fg{}\footverse{{Mt}(21:9,15)}
ni Simon-Pierre qui tombe à ses pieds, ni la femme pécheresse qui les embrasse,\footverse{{Lc}(7:38)}
ni l’aveugle guéri qui l’adore,\footverse{{Mc}(10:46), \ibibleverse{Lc}(18:38)}
ni l’apôtre Thomas qui lui dit\frcolon{} \Og{} Mon Seigneur et mon Dieu!\Fg{}\footverse{{Jn}(20:28)}
Certes, mes frères, dans toutes ces occasions, Jésus parle lorsqu’il se tait, et quand il accepte de tels hommages, c’est absolument comme s’il les réclamait lui-même.

Pour que Jésus prenne une telle attitude vis-à-vis des hommes, il faut qu’il soit et qu’il se sache dans un rapport unique avec Dieu. Il ne nous a pas laissés dans l’ignorance sur ce sublime sujet. Comme le nom qui définit ses rapports avec les hommes est celui de Fils de l’homme, le nom qui définit ses rapports avec Dieu est celui de Fils de Dieu. Jésus, sans doute, appelle ses disciples à devenir fils de Dieu, mais il se nomme \Og{} le Fils\Fg{} dans un sens excellent et unique, et parlant de Dieu, il ne dit pas \Og{} notre Père\Fg{}, il dit\frcolon{} \Og{} Mon Père et votre Père.\Fg{}\footverse{{Jn}(20:17)}
Fils de Dieu, Jésus est subordonné à son Père et dépendant de lui;
il ne possède rien qu’il n’ait reçu du Père;\footverse{{Jn}(5:19)}
il fait les œuvres que le Père lui a montrées, il dit les paroles que le Père lui a dites;
il cherche la gloire du Père, non la sienne propre, il dit\frcolon{} \Og{} Le Père est plus grand que moi.\Fg{}\footverse{{Jn}(14:28)}
Mais aussi le Père lui a tout dit, tout montré, tout donné.
En qualité de Fils, il participe à toutes les perfections du Père, à sa science, à sa puissance, à sa vie;
la gloire de Dieu est sa gloire, les anges de Dieu sont ses anges, l’Esprit de Dieu est son esprit.
À côté des passages où la subordination du Fils au Père est accentuée, se placent d’autres paroles où toute inégalité entre le Père et le Fils semble disparaître dans l’unité absolue, dans la réciprocité parfaite de l’amour\frcolon{} \Og{} Je suis dans le Père et le Père est en moi.\footverse{{Jn}(14:11)}
Nul ne connaît le Fils que le Père et nul ne connaît le Père que le Fils.\footverse{{Lc}(10:22)}
Tout ce qui est à moi est à toi; et tout ce qui est à toi est à moi.\footverse{{Jn}(17:10)}
Moi et le Père nous sommes un.\Fg{}\footverse{{Jn}(10:30)} \ocadr{} \Og{} Je sais d’où je viens et où je vais\Fg{},\footverse{{Jn}(8:14)} dit Jésus dans mon texte.
\ocadr{} D’où viens-tu, Seigneur? \ocadr{} \Og{} Je suis d’en haut,\footverse{{Jn}(8:23)} je suis descendu du ciel;\footverse{{Jn}(6:38,51)} avant qu’Abraham fût, je suis.\Fg{}\footverse{{Jn}(8:58)}
\ocadr{} Où vas-tu? \ocadr{} \Og{} Père, je vais à toi; glorifie-moi de la gloire que j’ai eue auprès de toi, avant que le monde fût.\Fg{}\footverse{{Jn}(17:5)}

Tels sont quelques-uns des témoignages que Jésus a rendus touchant sa personne. On peut les résumer en peu de mots\frcolon{} Jésus se donne comme étant, dans le sens le plus élevé qu’il soit possible d’attacher à ses termes, le Saint de Dieu, le Sauveur du monde, le Médiateur entre Dieu et les hommes, le Fils de Dieu. Je m’arrête là. Je n’invoque aujourd’hui ni le témoignage de l’Église, ni celui des apôtres eux-mêmes. Je vous conduis tout droit et exclusivement à la source même de la vérité chrétienne, le cœur et la conscience de Jésus. Si vous recevez le témoignage de Jésus-Christ, vous pouvez être embarrassé pour en concilier les divers éléments et les réunir sans les altérer dans une conception tout ensemble précise, complète, harmonique ; mais enfin, vous êtes au clair sur le point essentiel, vous savez qui est Jésus-Christ, et ce qu’il est venu faire dans le monde. Si vous recevez le témoignage de Jésus-Christ, il peut y avoir encore à vos yeux plus d’un mystère non éclairci, plus d’une formule consacrée qui appelle une révision ; mais enfin, la grande question qui divise notre Église est résolue pour vous, et elle est résolue, je l’affirme hardiment, dans le sens de la doctrine évangélique et du christianisme surnaturel. Si vous recevez le témoignage de Jésus-Christ, vous avez besoin encore, et plus que jamais, que l’Esprit de Dieu vous explique et vous approprie la vérité écrite dans le livre, qu’il glorifie en vous votre Sauveur ; mais enfin vous n’êtes plus à la merci de tout vent de doctrine, vous avez trouvé le roc inébranlable sur lequel doit s’édifier votre foi comme la foi de l’Église.

Mais une crainte vous reste. Ce rocher lui-même ne serait-il pas miné à son tour par l’effort des siècles, par le flot montant de l’incrédulité ? En d’autres termes, le témoignage de Jésus-Christ est-il vraiment inattaquable ? N’y a-t-il rien qui doive nous troubler dans cette objection en apparence si plausible des pharisiens\frcolon{} \Og{} Tu rends témoignage de toi-même, ton témoignage n’est pas véritable\Fg{}? C’est ce qui nous reste à examiner.

