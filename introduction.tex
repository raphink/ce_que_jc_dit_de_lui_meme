\addchap{}

{\chapfont\Large\scshape{\Huge M}es frères,\footnote{Sermon prêché le 5 juin 1872 dans le temple de l’Oratoire, à Paris, à l’occasion de l'ouverture du XXX\up{e} synode général de l'Église réformée de France.}}

C’est le 10 janvier 1660 que se dispersa le dernier synode national des Églises réformées de France ; le dernier du moins qui ait été tenu avec l’autorisation des pouvoirs publics, le synode de Loudun. À peine réuni, il avait eu la douleur d’apprendre, de la bouche du commissaire royal, que Sa Majesté (Louis XIV) avait résolu \Og{} que l’on n’assemblerait plus de synodes nationaux que lorsqu’elle le jugerait expédient\Fg{}. Vainement le synode protesta de toutes ses forces, par l’organe de son modérateur, Jean Daillé, qui alla jusqu’à dire\frcolon{} \Og{} Il est entièrement impossible que notre religion puisse se conserver sans tenir de ces sortes d’assemblées.\Fg{} S’il eût été sincère, le gouvernement aurait répondu qu’il partageait cette opinion, et que la destruction de la religion \Og{} prétendue réformée\Fg{} était précisément son but.

Les synodes du Désert, si grands par leur fidélité, par leur courage, par leur sagesse, ne furent pas, dans la rigueur du terme, des synodes nationaux. Le dernier de ces synodes se réunit dans le bas Languedoc, en~1763 ; son acte le plus saillant fut un engagement solennel par lequel les députés s’obligèrent à l’unanimité à maintenir de tout leur pouvoir, dans les Églises, l’unité de foi, de culte, de morale et de discipline.

Demain, 6~juin~1872, l’Église réformée de France rentre en possession de son régime synodal, sous la protection du gouvernement de la République française, qui s’est montré aussi jaloux de respecter nos libertés, et même aussi empressé à nous les rendre, que le despotisme de Louis~XIV était acharné à nous les ravir.

Gloire à Dieu, qui a eu compassion de notre Église, tout en l’éprouvant, et qui n’a pas permis qu’elle fondît tout entière dans le feu de la persécution la plus implacable, la plus systématique, la plus opiniâtre qui fut jamais ! Gloire à Dieu, qui la visite aujourd’hui dans sa miséricorde, et qui, en lui rendant l’institution qui est tout ensemble la garantie de son unité et la sauvegarde de ses libertés, l’appelle à entrer dans une ère nouvelle d’activité et de vie !

Et pourtant, avouons-le, mes frères\frcolon{} dans ce jour, dont la seule perspective eût fait bondir de joie le cœur de nos pères, des pensées de tristesse, d’inquiétude, ne sont étrangères à aucun membre du synode ; elles dominent chez quelques-uns. J’en dirai tout de suite la cause, attendu qu’il n’est ni bon, ni utile, ni possible de la dissimuler. Ce qui a fait la grandeur de notre Église, c’est sa foi. Cette foi, le premier de nos synodes, celui de~1559, à Paris, l’exprima d’un accord unanime dans une admirable confession qu’il écrivit en quelque sorte avec le sang des martyrs, au pied des échafauds et à la lueur des bûchers, et qu’on appela plus tard \Og{} la Confession de La Rochelle\Fg{}. À partir de ce jour, chaque nouveau synode national, au début de ses séances, lisait, sanctionnait, signait la confession de foi. Aujourd’hui nos Églises, privées depuis deux siècles de gouvernement central et régulier, n’ayant d’ailleurs ni pu ni voulu rester étrangères au mouvement de la pensée humaine durant ce long intervalle, ne sont plus unies dans la foi, comme elles l’étaient jadis. Étrange et cruelle situation ! Nous n’avons plus rien à craindre du gouvernement ni de la société qui nous entoure, mais nos divisions intérieures nous font plus de mal que la persécution elle-même n’a jamais pu nous en faire. Nous avons toute liberté pour affirmer, pour prêcher, dans une grande mesure pour propager nos croyances ; mais, comme Église, nous ne savons plus exactement ce que nous croyons. Membres du synode de~1872, que ferons-nous de cette question suprême, la question de la foi ? L’écarter par une fin de non-recevoir, n’est-ce pas décider et déclarer au monde que l’Église réformée n’est plus une Église, c’est-à-dire une société d’hommes attachés aux mêmes principes religieux, mais une association factice, un souvenir respectable, une \emph{expression historique}, je ne veux pas dire un chapitre au budget ? Tenter de résoudre cette question, n’est-ce pas nous diviser ? n’est-ce pas provoquer un schisme ?

Mes frères, ne craignez pas que je me permette d’anticiper sur les résolutions du synode ou même sur ses délibérations. Je me souviens que c’est en qualité de prédicateur de l’Évangile que je suis dans cette chaire, et je veux porter votre attention et la mienne plus haut que le terrain ecclésiastique. La question qui nous divise\frcolon{} \Og{} Quelle est ou quelle doit être la base religieuse de notre Église ?\Fg{} dépend évidemment de cette question plus générale\frcolon{} \Og{} Quel est le fondement même de la foi chrétienne ?\Fg{} Car, en qualité de chrétiens protestants, nous voulons tous retenir le fondement ; mais lorsqu’il s’agit de bâtir ceci ou cela sur le fondement, nous réclamons pour nous-mêmes et nous accordons aux autres la liberté.
Or, mes frères, à la question que je viens de formuler, je pense que nous serons tous d’accord pour répondre avec saint Paul\frcolon{} \Og{} Nul ne peut poser d’autre fondement que celui qui a été posé, à savoir, Jésus-Christ.\Fg{}\footverse{{ICo}(3:11)}
Mais peut-être n’apercevons-nous pas au premier abord toutes les conséquences de ce principe. La foi chrétienne est la foi en Jésus-Christ, c’est une proposition évidente par elle-même. Jésus-Christ est donc l’objet de la foi ; mais s’il en est l’objet, il en est aussi la raison et la règle ; autrement il n’en serait pas vraiment et souverainement l’objet. Le chrétien catholique romain croit en Jésus-Christ à cause de l’Église et conformément à ce que l’Église enseigne de lui\frcolon{} ce qui revient, dans la pratique, à mettre l’autorité de l’Église, c’est-à-dire aujourd’hui celle du pape, au-dessus de celle de Jésus-Christ. Le chrétien rationaliste croit en Jésus-Christ à cause de sa propre raison, et selon ce que sa raison admet ou conçoit\frcolon{} c’est-à-dire qu’en fait il met sa raison et sa conscience personnelles au-dessus de Jésus-Christ. Le chrétien évangélique, je veux dire le chrétien complet et conséquent dans sa foi, croit en Jésus-Christ à cause de Jésus-Christ et selon ce que Jésus-Christ a dit et pensé de lui-même. Le témoignage que Jésus-Christ a rendu de sa propre personne, est tout ensemble le dernier fondement et la règle suprême de sa foi. Le Seigneur Jésus, comme il le dit lui-même dans les paroles qui précèdent immédiatement notre texte, est la lumière du monde; or, la lumière se prouve en se montrant et n’a pas besoin pour être aperçue d’une autre clarté que la sienne.
\quotebox{\Og{} Je suis la lumière du monde. Celui qui me suit ne marchera pas dans les ténèbres, mais il aura au contraire la lumière de la vie. \Fg{}}{\ibibleverse{Jn}(8:12b)}
C’est ce témoignage de Jésus-Christ, fondement inébranlable de la foi du chrétien et de la foi de l’Église, dont je me propose de montrer aujourd’hui la nature et la force. Un tel sujet, tout en nous élevant au-dessus de nos tristes débats ecclésiastiques, me paraît profondément actuel, dans le meilleur sens du mot ; il a de plus un autre avantage, inappréciable à mes yeux, celui de reléguer dans l’ombre la personne et les opinions particulières du prédicateur. Il m’en a coûté, croyez-le bien, d’accepter l’honneur immérité qui m’était offert et de m’adresser, dans une circonstance si solennelle, à une assemblée où j’aperçois en grand nombre mes supérieurs par l’intelligence et par la science, mes aînés dans la vie, dans le ministère et dans la foi. Pourtant j’ose vous dire comme Etienne\frcolon{} \Og{} Mes frères et mes pères, écoutez-moi\Fg{}; car ce n’est pas pour les pensées d’un homme pécheur que je réclame votre attention, c’est pour Jésus-Christ lui-même vous parlant de Jésus-Christ.


